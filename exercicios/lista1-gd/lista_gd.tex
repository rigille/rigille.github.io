\begin{itemize}
\tightlist
\item
  \href{/blog}{blog}
\item
  \href{/exercicios}{exercicios}
\item
  \href{/}{sobre}
\end{itemize}

\hypertarget{post__title}{%
\section{Lista 1 --- Geometria Diferencial}\label{post__title}}

Apr 10 2021

\hypertarget{capuxedtulo-1--seuxe7uxe3o-3}{%
\subsection{\texorpdfstring{Capítulo 1 --- Seção
3\protect\hyperlink{capuxedtulo-1--seuxe7uxe3o-3}{\#}}{Capítulo 1 --- Seção 3\#}}\label{capuxedtulo-1--seuxe7uxe3o-3}}

\hypertarget{questuxe3o-8-}{%
\subsubsection{\texorpdfstring{Questão 8
🖋️\protect\hyperlink{questuxe3o-8-}{\#}}{Questão 8 🖋️\#}}\label{questuxe3o-8-}}

Seja \$\textbackslash{}alpha : I \textbackslash{}rightarrow R\^{}3\$ uma
curva diferenciável e seja \${[}a, b{]} \textbackslash{}subset I\$ ser
um intervalo fechado. Pra cada partição

\$\$a = t\_0 \textless{} t\_1 \textless{} \ldots{} \textless{} t\_n =
b\$\$

de \${[}a, b{]}\$, considere a soma \$\textbackslash{}sum\_\{i=1\}\^{}n
\textbar{}\textbackslash{}alpha(t\_i) -
\textbackslash{}alpha(t\_\{i-1\})\textbar{} = I(\textbackslash{}alpha,
P)\$ onde \$P\$ representa a partição. A norma \$\textbar{}P\textbar{}\$
de uma partição \$\textbar{}P\textbar{}\$ é definida como

\$\$\textbar{}P\textbar{} = \textbackslash{}mathrm\{max\}(t\_i -
t\_\{i-1\}),\textbackslash{} i = 1, \ldots{}, n\$\$

Geometricamente, \$I(\textbackslash{}alpha, P)\$ é o comprimento de um
polígono inscrito em \$\textbackslash{}alpha({[}a, b{]})\$ com vértices
em \$\textbackslash{}alpha(t\_i)\$. O objetivo deste exercício é mostrar
que o comprimento de arco de \$\textbackslash{}alpha({[}a, b{]})\$ é, em
certo sentido, um limite dos comprimentos dos polígonos inscritos.

Prove que, dado um \$\textbackslash{}varepsilon \textgreater{} 0\$,
existe um \$\textbackslash{}delta \textgreater{} 0\$ tal que
\$\textbar{}P\textbar{} \textless{} \textbackslash{}delta\$ então

\$\$\textbackslash{}left\textbar{}\textbackslash{}int\_a\^{}b
\textbar{}\textbackslash{}alpha'(t)\textbar{}\textbackslash{}mathrm\{d\}t
- I(\textbackslash{}alpha, P)\textbackslash{}right\textbar{} \textless{}
\textbackslash{}varepsilon\$\$

\hypertarget{resoluuxe7uxe3o}{%
\paragraph{Resolução}\label{resoluuxe7uxe3o}}

Primeiro consideremos que

\$\$I(\textbackslash{}alpha, P) =
\textbackslash{}sum\_\{i=0\}\^{}\{n-1\} \textbackslash{}left\textbar{}
\textbackslash{}int\_\{t\_i\}\^{}\{t\_\{i+1\}\}
\textbackslash{}alpha'(t) \textbackslash{}mathrm\{d\}t
\textbackslash{}right\textbar{} \textbackslash{}leq
\textbackslash{}sum\_\{i=0\}\^{}\{n-1\}
\textbackslash{}int\_\{t\_i\}\^{}\{t\_\{i+1\}\}
\textbar{}\textbackslash{}alpha'(t)\textbar{}
\textbackslash{}mathrm\{d\}t = \textbackslash{}int\_a\^{}b
\textbar{}\textbackslash{}alpha'(t)\textbar{}
\textbackslash{}mathrm\{d\}t \textbackslash{}tag\{0\}\$\$

Como \$\textbackslash{}alpha'\$ é contínua no intervalo compacto \${[}a,
b{]}\$, ela é uniformemente contínua. Isto significa que, dado
\$\textbackslash{}varepsilon \textgreater{} 0\$, existe
\$\textbackslash{}delta \textgreater{} 0\$ tal que

\$\$\textbar{}\textbackslash{}alpha'(s) -
\textbackslash{}alpha'(t)\textbar{} \textless{}
\textbackslash{}varepsilon\$\$

sempre que \$\textbar{}s - t\textbar{} \textless{}
\textbackslash{}delta\$. Em particular, se \$\textbar{}P\textbar{}
\textless{} \textbackslash{}delta\$ e \$t\_i \textless{} k \textless{}
t\_\{i+1\}\$

\$\$\textbar{}\textbackslash{}alpha'(k)\textbar{} \textbackslash{}leq
\textbar{}\textbackslash{}alpha'(t\_\{i+1\})\textbar{} +
\textbackslash{}varepsilon\$\$

daí que

\$\$\textbackslash{}int\_\{t\_i\}\^{}\{t\_\{i+1\}\}
\textbar{}\textbackslash{}alpha'(t)\textbar{}
\textbackslash{}mathrm\{d\}t \textbackslash{}leq
\textbar{}\textbackslash{}alpha'(t\_\{i+1\})\textbar{}(t\_\{i+1\} -
t\_i) + \textbackslash{}varepsilon (t\_\{i+1\} - t\_i)\$\$

\$\$= \textbackslash{}int\_\{t\_i\}\^{}\{t\_\{i+1\}\}
\textbar{}\textbackslash{}alpha'(t) + \textbackslash{}alpha'(t\_\{i+1\})
- \textbackslash{}alpha'(t)\textbar{} \textbackslash{}mathrm\{d\}t +
\textbackslash{}varepsilon (t\_\{i+1\} - t\_i)\$\$

\$\$\textbackslash{}leq
\textbackslash{}left\textbar{}\textbackslash{}int\_\{t\_i\}\^{}\{t\_\{i+1\}\}
\textbackslash{}alpha'(t) \textbackslash{}mathrm\{d\}t
\textbackslash{}right\textbar{} +
\textbackslash{}int\_\{t\_i\}\^{}\{t\_\{i+1\}\}
\textbar{}\textbackslash{}alpha'(t\_\{i+1\}) -
\textbackslash{}alpha'(t)\textbar{} \textbackslash{}mathrm\{d\}t +
\textbackslash{}varepsilon (t\_\{i+1\} - t\_i)\$\$
\$\$\textbackslash{}leq
\textbackslash{}left\textbar{}\textbackslash{}int\_\{t\_i\}\^{}\{t\_\{i+1\}\}
\textbackslash{}alpha'(t) \textbackslash{}mathrm\{d\}t
\textbackslash{}right\textbar{} + 2 \textbackslash{}varepsilon
(t\_\{i+1\} - t\_i)\$\$

ou seja

\$\$\textbackslash{}int\_\{t\_i\}\^{}\{t\_\{i+1\}\}
\textbar{}\textbackslash{}alpha'(t)\textbar{}
\textbackslash{}mathrm\{d\}t \textbackslash{}leq
\textbackslash{}left\textbar{}\textbackslash{}int\_\{t\_i\}\^{}\{t\_\{i+1\}\}
\textbackslash{}alpha'(t) \textbackslash{}mathrm\{d\}t
\textbackslash{}right\textbar{} + 2 \textbackslash{}varepsilon
(t\_\{i+1\} - t\_i)\$\$

\$\$\textbackslash{}sum\_\{i=0\}\^{}\{n-1\}
\textbackslash{}int\_\{t\_i\}\^{}\{t\_\{i+1\}\}
\textbar{}\textbackslash{}alpha'(t)\textbar{}
\textbackslash{}mathrm\{d\}t \textbackslash{}leq
\textbackslash{}sum\_\{i=0\}\^{}\{n-1\}\textbackslash{}left\textbar{}\textbackslash{}int\_\{t\_i\}\^{}\{t\_\{i+1\}\}
\textbackslash{}alpha'(t) \textbackslash{}mathrm\{d\}t
\textbackslash{}right\textbar{} + 2 \textbackslash{}varepsilon
(t\_\{i+1\} - t\_i)\$\$

\$\$\textbackslash{}int\_a\^{}b
\textbar{}\textbackslash{}alpha'(t)\textbar{}
\textbackslash{}mathrm\{d\}t \textbackslash{}leq
I(\textbackslash{}alpha, P) + 2 \textbackslash{}varepsilon (b - a)\$\$

por fim, juntando \$(0)\$ e a linha de cima

\$\$0 \textbackslash{}leq \textbackslash{}int\_a\^{}b
\textbar{}\textbackslash{}alpha'(t)\textbar{}
\textbackslash{}mathrm\{d\}t - I(\textbackslash{}alpha, P)
\textbackslash{}leq 2 \textbackslash{}varepsilon (b - a)\$\$

como o \$\textbackslash{}varepsilon\$ é genérico, isso conclui a
demonstração

\hypertarget{questuxe3o-9-}{%
\subsubsection{\texorpdfstring{Questão 9
🖋️\protect\hyperlink{questuxe3o-9-}{\#}}{Questão 9 🖋️\#}}\label{questuxe3o-9-}}

\textbf{a)} Seja \$\textbackslash{}alpha: I \textbackslash{}rightarrow
R\^{}3\$ uma curva de classe \$C\^{}0\$. Use a aproximação de polígonos
descrita no Exercício 8 para dar uma definição razoável do comprimento
de arco de \$\textbackslash{}alpha\$.

\hypertarget{resoluuxe7uxe3o-1}{%
\paragraph{Resolução}\label{resoluuxe7uxe3o-1}}

No caso \$C\^{}0\$, não faz sentido falar em integrar a derivada, ela
pode não existir. Mas continua fazendo todo sentido falar no comprimento
da poligonal inscrita \$I(\textbackslash{}alpha, P)\$. Assim seria bem
razoável definir o comprimento de \$\textbackslash{}alpha\$ como

\$\$\textbackslash{}lim\_\{\textbar{}P\textbar{}
\textbackslash{}rightarrow 0\} I(\textbackslash{}alpha, P)\$\$

\textbf{b)} (Uma curva não-retificável) O exemplo a seguir mostra que,
com qualquer definição razoável, o comprimento de uma curva pode ser
ilimitado. Seja \$\textbackslash{}alpha: {[}0, 1{]}
\textbackslash{}rightarrow R\^{}2\$ dado como \$\textbackslash{}alpha(t)
= (t, t \textbackslash{}mathrm\{sin\}(\textbackslash{}pi/t))\$ se \$t
\textbackslash{}neq 0\$, e \$\textbackslash{}alpha(0) = (0, 0)\$.
Mostre, geometricamente que o comprimento de arco da porção da curva
correspondendo a \$1/(n+1) \textbackslash{}leq t \textbackslash{}leq
1/n\$ é pelo menos
\$2/\textbackslash{}left(n+\textbackslash{}frac\{1\}\{2\}\textbackslash{}right)\$.
Use isso para mostrar que o comprimento da curva no intevalo \$1/N
\textbackslash{}leq t \textbackslash{}leq 1\$ é maior que \$2
\textbackslash{}sum\_\{n=1\}\^{}N 1/(n+1)\$ e portanto que ele tende a
infinito conforme \$N \textbackslash{}rightarrow
\textbackslash{}infty\$.

Usaremos a poligonal que passa pelos pontos

\$\$P\_0 = (1/(n+1), 0)\$\$ \$\$P\_1 = (1/(n+1/2), 1/(n+1/2)\$\$
\$\$P\_2 = (1/n, 0)\$\$

para conseguir a nossa cota inferior de comprimento no intervalo.
Ficaria

\$\$ \textbar{}P\_1 - P\_0\textbar{} + \textbar{}P\_1 - P\_2\textbar{}
\textbackslash{}leq \textbackslash{}frac\{2\}\{n + 1/2\} \$\$

e, de fato, usando essa estimativa

\$\$\textbackslash{}int\_\{1/N\}\^{}1
\textbar{}\textbackslash{}alpha'(t)\textbar{}\textbackslash{}mathrm\{d\}t
= \textbackslash{}sum\_\{k = 1\}\^{}\{N-1\}
\textbackslash{}int\_\{1/(k+1)\}\^{}\{1/k\}
\textbar{}\textbackslash{}alpha'(t)\textbar{}
\textbackslash{}mathrm\{d\}t \textbackslash{}leq
\textbackslash{}sum\_\{k = 1\}\^{}\{N-1\} \textbackslash{}frac\{2\}\{n +
1/2\}\$\$

A série da direita pode ser comparada com a série harmônica

\$\$\textbackslash{}sum\_\{k = 1\}\^{}\{\textbackslash{}infty\}
\textbackslash{}frac\{2\}\{n + 1/2\} \textbackslash{}geq
\textbackslash{}sum\_\{k = 1\}\^{}\{\textbackslash{}infty\}
\textbackslash{}frac\{1\}\{n\} = +\textbackslash{}infty\$\$

A curva não é retificável.

\hypertarget{questuxe3o-10-}{%
\subsubsection{\texorpdfstring{Questão 10
🖋️\protect\hyperlink{questuxe3o-10-}{\#}}{Questão 10 🖋️\#}}\label{questuxe3o-10-}}

(Linhas retas como as mais curtas) Seja \$\textbackslash{}alpha: I
\textbackslash{}rightarrow R\^{}3\$ uma curva parametrizada. Seja
\${[}a, b{]} \textbackslash{}subset I\$ e fixe
\$\textbackslash{}alpha(a) = p\$, \$\textbackslash{}alpha(b) = q\$.

\textbf{a)} Mostre que, para qualquer vetor constante \$v\$, com
\$\textbar{}v\textbar{} = 1\$,

\$\$(q - p) \textbackslash{}cdot v = \textbackslash{}int\_a\^{}b
\textbackslash{}alpha'(t) \textbackslash{}cdot v
\textbackslash{}mathrm\{d\}t \textbackslash{}leq
\textbackslash{}int\_a\^{}b
\textbar{}\textbackslash{}alpha'(t)\textbar{}
\textbackslash{}mathrm\{d\}t\$\$

\hypertarget{resoluuxe7uxe3o-2}{%
\paragraph{Resolução}\label{resoluuxe7uxe3o-2}}

Pelo teorema fundamental do cálculo

\$\$q - p = \textbackslash{}int\_a\^{}b \textbackslash{}alpha'(t)
\textbackslash{}mathrm\{d\}t\$\$

tirando o produto interno de ambos os lados

\$\$\textbackslash{}langle q - p, v \textbackslash{}rangle =
\textbackslash{}left\textbackslash{}langle \textbackslash{}int\_a\^{}b
\textbackslash{}alpha'(t) \textbackslash{}mathrm\{d\}t, v
\textbackslash{}right\textbackslash{}rangle\$\$

pela linearidade da integral

\$\$\textbackslash{}langle q - p, v \textbackslash{}rangle =
\textbackslash{}int\_a\^{}b \textbackslash{}left\textbackslash{}langle
\textbackslash{}alpha'(t), v \textbackslash{}right\textbackslash{}rangle
\textbackslash{}mathrm\{d\}t \$\$

para fechar, pela desigualdade de cauchy-schwarz

\$\$\textbackslash{}int\_a\^{}b
\textbackslash{}left\textbackslash{}langle \textbackslash{}alpha'(t), v
\textbackslash{}right\textbackslash{}rangle \textbackslash{}mathrm\{d\}t
\textbackslash{}leq \textbackslash{}int\_a\^{}b \textbar{}
\textbackslash{}alpha'(t)\textbar{} \textbackslash{}cdot \textbar{} v
\textbar{} \textbackslash{}mathrm\{d\}t = \textbackslash{}int\_a\^{}b
\textbar{} \textbackslash{}alpha'(t)\textbar{}
\textbackslash{}mathrm\{d\}t\$\$

\textbf{b)} Fixe

\$\$v = \textbackslash{}frac\{q-p\}\{\textbar{}q-p\textbar{}\}\$\$

e mostre que

\$\$\textbar{}\textbackslash{}alpha(b) -
\textbackslash{}alpha(a)\textbar{} \textbackslash{}leq
\textbackslash{}int\_a\^{}b
\textbar{}\textbackslash{}alpha'(t)\textbar{}
\textbackslash{}mathrm\{d\}t\$\$

Isto é, que a curva de menor comprimento ligando
\$\textbackslash{}alpha(a)\$ e \$\textbackslash{}alpha(b)\$ é a linha
reta ligando esses pontos.

\hypertarget{resoluuxe7uxe3o-3}{%
\paragraph{Resolução}\label{resoluuxe7uxe3o-3}}

Como \$v\$ é unitário, vale a identidade da letra anterior. Substituindo

\$\$\textbackslash{}left\textbackslash{}langle \textbackslash{}alpha(b)
- \textbackslash{}alpha(a),
\textbackslash{}frac\{\textbackslash{}alpha(b)-\textbackslash{}alpha(a)\}\{\textbar{}\textbackslash{}alpha(b)-\textbackslash{}alpha(a)\textbar{}\}
\textbackslash{}right\textbackslash{}rangle \textbackslash{}leq
\textbackslash{}int\_a\^{}b \textbar{}
\textbackslash{}alpha'(t)\textbar{} \textbackslash{}mathrm\{d\}t\$\$

\$\$\textbar{}\textbackslash{}alpha(b)-\textbackslash{}alpha(a)\textbar{}
\textbackslash{}leq \textbackslash{}int\_a\^{}b \textbar{}
\textbackslash{}alpha'(t)\textbar{} \textbackslash{}mathrm\{d\}t\$\$

\hypertarget{capuxedtulo-1--seuxe7uxe3o-5}{%
\subsection{\texorpdfstring{Capítulo 1 --- Seção
5\protect\hyperlink{capuxedtulo-1--seuxe7uxe3o-5}{\#}}{Capítulo 1 --- Seção 5\#}}\label{capuxedtulo-1--seuxe7uxe3o-5}}

\hypertarget{questuxe3o-4-}{%
\subsubsection{\texorpdfstring{Questão 4
🖋️\protect\hyperlink{questuxe3o-4-}{\#}}{Questão 4 🖋️\#}}\label{questuxe3o-4-}}

Assuma que todas as normais de uma curva parametrizada passam por um
ponto fixado, prove que a curva está contida em um círculo.

\hypertarget{resoluuxe7uxe3o-4}{%
\paragraph{Resolução}\label{resoluuxe7uxe3o-4}}

Sem perda de generalidade, vamos assumir que o ponto fixado é a origem,
do contrário poderíamos aplicar um movimento rígido para movimentar o
ponto para origem. Dizer que as normais passam pela origem é a mesma
coisa que dizer que o vetor posição e o vetor tangente são ortogonais.
Então:

\$\$ \textbackslash{}langle \textbackslash{}alpha(t),
\textbackslash{}alpha'(t) \textbackslash{}rangle = 0\$\$

Sabendo dessa identidade, vamos analizar a derivada do quadrado da norma
do vetor posição

\$\$\textbackslash{}frac\{\textbackslash{}mathrm\{d\}\}\{\textbackslash{}mathrm\{d\}t\}
\textbackslash{}langle \textbackslash{}alpha(t),
\textbackslash{}alpha(t) \textbackslash{}rangle\$\$

\$\$\textbackslash{}langle \textbackslash{}alpha(t),
\textbackslash{}alpha'(t) \textbackslash{}rangle +
\textbackslash{}langle \textbackslash{}alpha'(t),
\textbackslash{}alpha(t) \textbackslash{}rangle\$\$

\$\$0\$\$

Ahá! Então a derivada é zero. Daí concluímos que a norma é constante e
que a curva está contida no círculo.

\hypertarget{questuxe3o-6-}{%
\subsubsection{\texorpdfstring{Questão 6
🖋️\protect\hyperlink{questuxe3o-6-}{\#}}{Questão 6 🖋️\#}}\label{questuxe3o-6-}}

Uma translação por um vetor \$v\$ em \$R\^{}3\$ é uma aplicação \$A:
R\^{}3 \textbackslash{}rightarrow R\^{}3\$ que é dado por \$A(p) = p +
v\$, \$p \textbackslash{}in R\^{}3\$. Uma aplicação linear
\$\textbackslash{}rho : R\^{}3 \textbackslash{}rightarrow R\^{}3\$ é uma
\emph{transformação ortogonal} quando \$\textbackslash{}rho u
\textbackslash{}cdot \textbackslash{}rho v = u \textbackslash{}cdot v\$
para todos os vetores \$u,\textbackslash{} v \textbackslash{}in
R\^{}3\$. Um \emph{movimento rígido} em \$R\^{}3\$ é o resultado de
compor uma translação com uma transformação ortogonal com determinante
positivo (esta última condição é incluída porque esperamos que
movimentos rígidos preservem orientações).

\textbf{a)} Demonstre que a norma de um vetor e o ângulo
\$\textbackslash{}theta\$ entre dois vetores, \$0 \textbackslash{}leq
\textbackslash{}theta \textbackslash{}leq \textbackslash{}pi\$, são
invariantes sob transformações ortogonais com determinante positivo.

\hypertarget{resoluuxe7uxe3o-5}{%
\paragraph{Resolução}\label{resoluuxe7uxe3o-5}}

Seja \$u \textbackslash{}in R\^{}3\$, vamos investigar \$\textbar{}
\textbackslash{}rho(u) \textbar{}\$:

\$\$\textbar{} \textbackslash{}rho(u) \textbar{}\$\$

\$\$\textbackslash{}sqrt\{\textbackslash{}langle \textbackslash{}rho(u),
\textbackslash{}rho(u) \textbackslash{}rangle\}\$\$

\$\$\textbackslash{}sqrt\{\textbackslash{}langle u, u
\textbackslash{}rangle\}\$\$

\$\$\textbar{} u \textbar{}\$\$

Agora sejam \$u,\textbackslash{} v \textbackslash{}in R\^{}3\$, o ângulo
\$\textbackslash{}theta\$ entre \$\textbackslash{}rho(u)\$ e
\$\textbackslash{}rho(v)\$ é:

\$\$\textbackslash{}theta =
\textbackslash{}mathrm\{arccos\}\textbackslash{}left(\textbackslash{}frac\{\textbackslash{}langle
\textbackslash{}rho(u), \textbackslash{}rho(v)
\textbackslash{}rangle\}\{\textbar{} \textbackslash{}rho(u) \textbar{}
\textbackslash{}cdot \textbar{} \textbackslash{}rho(v)
\textbar{}\}\textbackslash{}right)\$\$

simplificando

\$\$\textbackslash{}theta =
\textbackslash{}mathrm\{arccos\}\textbackslash{}left(\textbackslash{}frac\{\textbackslash{}langle
u, v \textbackslash{}rangle\}\{\textbar{} u \textbar{}
\textbackslash{}cdot \textbar{} v \textbar{}\}\textbackslash{}right)\$\$

que é o mesmo que o ângulo entre \$u\$ e \$v\$.

\textbf{b)} Mostre que o produto vetorial de dois vetores é invariante
sob transformações ortogonais com determinante positivo. Essa afirmação
ainda é válida se removermos a condição do determinante?

\hypertarget{resoluuxe7uxe3o-6}{%
\paragraph{Resolução}\label{resoluuxe7uxe3o-6}}

Essa é mais complicada. A ortogonalidade da transformação, por si só,
não é o bastante para determinar qual o sentido do produto vetorial.
Para conseguir usar a hipótese do determinante efetivamente, o que me
deixou surpreso, fica mais fácil considerar o caso mais geral. Primeiro
considere que

\$\$ \textbackslash{}langle u, v \textbackslash{}times w
\textbackslash{}rangle = \textbackslash{}mathrm\{det\}(u, v, w)\$\$

onde \$\textbackslash{}mathrm\{det\}(u, v, w)\$ é o determinante da
matriz que tem linhas \$u\$, \$v\$ e \$w\$.

\$\$ \textbackslash{}mathrm\{det\}(A)\textbackslash{}langle u, v
\textbackslash{}times w \textbackslash{}rangle =
\textbackslash{}mathrm\{det\}(A)\textbackslash{}mathrm\{det\}(u, v, w)
\$\$

\$\$ \textbackslash{}langle u, \textbackslash{}mathrm\{det\}(A) (v
\textbackslash{}times w) \textbackslash{}rangle =
\textbackslash{}mathrm\{det\}(A)\textbackslash{}mathrm\{det\}(u, v, w)
\$\$

\$\$ \textbackslash{}langle u, \textbackslash{}mathrm\{det\}(A) (v
\textbackslash{}times w) \textbackslash{}rangle =
\textbackslash{}mathrm\{det\}(Au, Av, Aw) \$\$

\$\$ \textbackslash{}langle u, \textbackslash{}mathrm\{det\}(A) (v
\textbackslash{}times w) \textbackslash{}rangle = \textbackslash{}langle
Au, Av \textbackslash{}times Aw \textbackslash{}rangle \$\$

\$\$ \textbackslash{}langle u, \textbackslash{}mathrm\{det\}(A) (v
\textbackslash{}times w) \textbackslash{}rangle = \textbackslash{}langle
u, A\^{}T (Av \textbackslash{}times Aw) \textbackslash{}rangle \$\$

como esta identidade vale para qualquer escolha de \$u\$, os dois
fatores do produto devem ser iguais.

\$\$\textbackslash{}mathrm\{det\}(A) (v \textbackslash{}times w) =
A\^{}T (Av \textbackslash{}times Aw)\$\$

no belíssimo caso em que \$A\^{}T\$ é invertível, isso é equivalente a

\$\$\textbackslash{}mathrm\{det\}(A) (A\^{}T)\^{}\{-1\} (v
\textbackslash{}times w) = Av \textbackslash{}times Aw\$\$

especializando \$A = \textbackslash{}rho\$

\$\$\textbackslash{}mathrm\{det\}(\textbackslash{}rho)
(\textbackslash{}rho\^{}T)\^{}\{-1\} (v \textbackslash{}times w) =
\textbackslash{}rho(v) \textbackslash{}times \textbackslash{}rho(w)\$\$

\$\$ \textbackslash{}rho (v \textbackslash{}times w) =
\textbackslash{}rho(v) \textbackslash{}times \textbackslash{}rho(w) \$\$

\textbf{c)} Mostre que o comprimento de arco, a curvatora e a torção de
uma curva parametrizada são (sempre que definidos) invariantes sob
movimentos rígidos.

Usaremos as identidades que mostramos e as identidades que mostraremos
no exercício 12 para demonstrar esta questão.

Comprimento de arco:

\$\$\textbackslash{}int\_a\^{}b \textbar{}(\textbackslash{}rho
\textbackslash{}circ \textbackslash{}alpha)'(t)\textbar{}
\textbackslash{}mathrm\{d\}t\$\$

\$\$\textbackslash{}int\_a\^{}b \textbar{}\textbackslash{}rho
\textbackslash{}cdot \textbackslash{}alpha'(t)\textbar{}
\textbackslash{}mathrm\{d\}t\$\$

\$\$\textbackslash{}int\_a\^{}b
\textbar{}\textbackslash{}alpha'(t)\textbar{}
\textbackslash{}mathrm\{d\}t\$\$

Curvatura:

\$\$\textbackslash{}frac\{\textbar{}(\textbackslash{}rho
\textbackslash{}circ \textbackslash{}alpha)' \textbackslash{}wedge
(\textbackslash{}rho \textbackslash{}circ
\textbackslash{}alpha)''\textbar{}\}\{\textbar{}\textbackslash{}rho
\textbackslash{}circ \textbackslash{}alpha'\textbar{}\^{}3\}\$\$

\$\$\textbackslash{}frac\{\textbar{}\textbackslash{}rho
\textbackslash{}cdot \textbackslash{}alpha' \textbackslash{}wedge
\textbackslash{}rho \textbackslash{}cdot
\textbackslash{}alpha''\textbar{}\}\{\textbar{}\textbackslash{}rho
\textbackslash{}cdot \textbackslash{}alpha'\textbar{}\^{}3\}\$\$

\$\$\textbackslash{}frac\{\textbar{}\textbackslash{}rho
\textbackslash{}cdot (\textbackslash{}alpha' \textbackslash{}wedge
\textbackslash{}alpha'')\textbar{}\}\{\textbar{}\textbackslash{}rho
\textbackslash{}cdot \textbackslash{}alpha'\textbar{}\^{}3\}\$\$

\$\$\textbackslash{}frac\{\textbar{}\textbackslash{}alpha'
\textbackslash{}wedge
\textbackslash{}alpha''\textbar{}\}\{\textbar{}\textbackslash{}alpha'\textbar{}\^{}3\}\$\$

Torção:

\$\$-\textbackslash{}frac\{((\textbackslash{}rho \textbackslash{}circ
\textbackslash{}alpha)' \textbackslash{}wedge (\textbackslash{}rho
\textbackslash{}circ \textbackslash{}alpha)'') \textbackslash{}cdot
(\textbackslash{}rho \textbackslash{}circ
\textbackslash{}alpha)'''\}\{\textbar{}(\textbackslash{}rho
\textbackslash{}circ \textbackslash{}alpha) \textbackslash{}wedge
(\textbackslash{}rho \textbackslash{}circ
\textbackslash{}alpha)''\textbar{}\^{}2\}\$\$

\$\$-\textbackslash{}frac\{(\textbackslash{}rho \textbackslash{}cdot
\textbackslash{}alpha' \textbackslash{}wedge \textbackslash{}rho
\textbackslash{}cdot \textbackslash{}alpha'') \textbackslash{}cdot
\textbackslash{}rho \textbackslash{}cdot
\textbackslash{}alpha'''\}\{\textbar{}\textbackslash{}rho
\textbackslash{}cdot \textbackslash{}alpha \textbackslash{}wedge
\textbackslash{}rho \textbackslash{}cdot
\textbackslash{}alpha''\textbar{}\^{}2\}\$\$

\$\$-\textbackslash{}frac\{(\textbackslash{}rho \textbackslash{}cdot
(\textbackslash{}alpha' \textbackslash{}wedge \textbackslash{}alpha''))
\textbackslash{}cdot \textbackslash{}rho \textbackslash{}cdot
\textbackslash{}alpha'''\}\{\textbar{}\textbackslash{}rho
\textbackslash{}cdot (\textbackslash{}alpha \textbackslash{}wedge
\textbackslash{}alpha'')\textbar{}\^{}2\}\$\$

\$\$-\textbackslash{}frac\{(\textbackslash{}alpha' \textbackslash{}wedge
\textbackslash{}alpha'') \textbackslash{}cdot
\textbackslash{}alpha'''\}\{\textbar{}\textbackslash{}alpha
\textbackslash{}wedge \textbackslash{}alpha''\textbar{}\^{}2\}\$\$

\hypertarget{questuxe3o-12}{%
\subsubsection{\texorpdfstring{Questão
12\protect\hyperlink{questuxe3o-12}{\#}}{Questão 12\#}}\label{questuxe3o-12}}

Seja \$\textbackslash{}alpha: I \textbackslash{}rightarrow R\^{}3\$ uma
curva regular parametrizada (não necessariamente por comprimento de
arco) e seja \$\textbackslash{}beta: J \textbackslash{}rightarrow
R\^{}3\$ uma parametrização de \$\textbackslash{}alpha(I)\$ pelo
comprimento de arco \$s = s(t)\$, medida de \$t\_0 \textbackslash{}in
I\$s. Seja \$t = t(s)\$ a função inversa de \$s\$ e fixe
\$\textbackslash{}mathrm\{d\}\textbackslash{}alpha/\textbackslash{}mathrm\{d\}t
= \textbackslash{}alpha'\$,
\$\textbackslash{}mathrm\{d\}\^{}2\textbackslash{}alpha/\textbackslash{}mathrm\{d\}t\^{}2
= \textbackslash{}alpha''\$, etc. Prove que

\textbf{a)} \$\textbackslash{}mathrm\{d\}t/\textbackslash{}mathrm\{d\}s
= 1/\textbar{}\textbackslash{}alpha'\textbar{}\$,
\$\textbackslash{}mathrm\{d\}\^{}2t/\textbackslash{}mathrm\{d\}s\^{}2 =
-(\textbackslash{}alpha'\textbackslash{}cdot
\textbackslash{}alpha''/\textbar{}\textbackslash{}alpha'\textbar{}\^{}4)\$.

\textbf{b)} A curvatora de \$\textbackslash{}alpha\$ em \$t
\textbackslash{}in I\$ é

\$\$k(t) =
\textbackslash{}frac\{\textbar{}\textbackslash{}alpha'\textbackslash{}wedge
\textbackslash{}alpha''\textbar{}\}\{\textbar{}\textbackslash{}alpha'\textbar{}\^{}3\}\$\$

\textbf{c)} A torção de \$\textbackslash{}alpha\$ em \$t
\textbackslash{}in I\$ é

\$\$\textbackslash{}tau(t) = -
\textbackslash{}frac\{(\textbackslash{}alpha' \textbackslash{}wedge
\textbackslash{}alpha'') \textbackslash{}cdot
\textbackslash{}alpha'''\}\{\textbar{}\textbackslash{}alpha
\textbackslash{}wedge \textbackslash{}alpha''\textbar{}\^{}2\}\$\$

\textbf{d)} Se \$\textbackslash{}alpha: I \textbackslash{}rightarrow
R\^{}2\$ é uma curva plana \$\textbackslash{}alpha(t) = (x(t), y(t))\$,
a curvatura com sinal de \$\textbackslash{}alpha\$ no \$t\$ é

\$\$k(t) = \textbackslash{}frac\{x'y'' - x'' y'\}\{\textbackslash{}left(
(x')\^{}2 + (y')\^{}2 \textbackslash{}right)\^{}\{3/2\}\}\$\$

\hypertarget{capuxedtulo-2--seuxe7uxe3o-2}{%
\subsection{\texorpdfstring{Capítulo 2 --- Seção
2\protect\hyperlink{capuxedtulo-2--seuxe7uxe3o-2}{\#}}{Capítulo 2 --- Seção 2\#}}\label{capuxedtulo-2--seuxe7uxe3o-2}}

\hypertarget{questuxe3o-16-}{%
\subsubsection{\texorpdfstring{Questão 16
🖋️\protect\hyperlink{questuxe3o-16-}{\#}}{Questão 16 🖋️\#}}\label{questuxe3o-16-}}

Uma forma de definir um sistema de coordenadas para a esfera \$S\^{}2\$,
dada por \$x\^{}2 + y\^{}2 + (z-1)\^{}2 = 1\$\$ é considerar a chamada
\emph{projeção estereográfica} \$\textbackslash{}pi: S\^{}2
\textbackslash{}setminus \{N\} \textbackslash{}rightarrow R\^{}2\$ que
carrega um ponto \$p = (x, y, z)\$ da esfera \$S\^{}2\$ menos o polo
norte \$N = (0, 0, 2)\$ na intersecção com o plano \$xy\$ com a linha
reta que conecta \$N\$ a \$p\$. Seja \$(u, v) = \textbackslash{}pi(x, y,
z)\$, onde \$(x, y, z) \textbackslash{}in S\^{}2
\textbackslash{}setminus \{N\}\$ e \$(u, v)\$ pertence ao plano \$xy\$.

\textbf{a)} Mostre que \$\textbackslash{}pi\^{}\{-1\}: R\^{}2
\textbackslash{}rightarrow S\^{}2\$ é dada por

\$\$\textbackslash{}begin\{cases\} x =
\textbackslash{}frac\{4u\}\{u\^{}2 + v\^{}2 + 4\}
\textbackslash{}\textbackslash{}\\
y = \textbackslash{}frac\{4v\}\{u\^{}2 + v\^{}2 + 4\}
\textbackslash{}\textbackslash{}\\
z = \textbackslash{}frac\{2(u\^{}2 + v\^{}2)\}\{u\^{}2 + v\^{}2 + 4\}
\textbackslash{}\textbackslash{}\\
\textbackslash{}end\{cases\}\$\$

\hypertarget{resoluuxe7uxe3o-7}{%
\paragraph{Resolução}\label{resoluuxe7uxe3o-7}}

Esta ilustração aqui ficou bonitinha, mas não vamos precisar dela

Para verificar se o ponto está ou não na esfera, vamos calcular

\$\$x\^{}2 + y\^{}2 + (z - 1)\^{}2\$\$

e verificar se isso dá \$1\$

\$\$\textbackslash{}left(\textbackslash{}frac\{4u\}\{u\^{}2 + v\^{}2 +
4\}\textbackslash{}right)\^{}2 +
\textbackslash{}left(\textbackslash{}frac\{4v\}\{u\^{}2 + v\^{}2 +
4\}\textbackslash{}right)\^{}2 +
\textbackslash{}left(\textbackslash{}frac\{2(u\^{}2 + v\^{}2)\}\{u\^{}2
+ v\^{}2 + 4\} - 1\textbackslash{}right)\^{}2\$\$

\$\$\textbackslash{}frac\{16u\^{}2 + 16v\^{}2 + (u\^{}2 + v\^{}2 -
4)\^{}2\}\{(u\^{}2 + v\^{}2 + 4)\^{}2\}\$\$

\$\$\textbackslash{}frac\{16u\^{}2 + 16v\^{}2 + (u\^{}2 + v\^{}2 -
4)\^{}2\}\{(u\^{}2 + v\^{}2 + 4)\^{}2\}\$\$

\$\$\textbackslash{}frac\{16u\^{}2 + 16v\^{}2 + u\^{}4 + v\^{}4 +
2u\^{}2v\^{}2 - 8u\^{}2 - 8v\^{}2 + 16\}\{(u\^{}2 + v\^{}2 +
4)\^{}2\}\$\$

\$\$\textbackslash{}frac\{8u\^{}2 + 8v\^{}2 + u\^{}4 + v\^{}4 +
2u\^{}2v\^{}2 + 16\}\{(u\^{}2 + v\^{}2 + 4)\^{}2\}\$\$

\$\$1\$\$

Bacana, então o ponto está na esfera. Agora resta verificar se \$D = (0,
0, 4)\$, \$E = (x, y, z)\$ e \$C = (u, v, 0)\$ são colineares. Se forem
não resta dúvida que essa é a expressão correta para
\$\textbackslash{}pi\^{}\{-1\}\$. De fato

\$\$C - D = (u, v, -4)\$\$

e

\$\$E - D = \textbackslash{}left(\textbackslash{}frac\{4\}\{u\^{}2 +
v\^{}2 + 4\}u, \textbackslash{}frac\{4\}\{u\^{}2 + v\^{}2 + 4\}v,
\textbackslash{}frac\{4\}\{u\^{}2 + v\^{}2 + 4\}\textbackslash{}cdot
(-4)\textbackslash{}right)\$\$

\textbf{b)} Mostre que é possível, usando projeção estereográfica,
cobrir a esfera com duas vizinhanças coordenadas.

\hypertarget{resoluuxe7uxe3o-8}{%
\paragraph{Resolução}\label{resoluuxe7uxe3o-8}}

Se você definir a transformação afim \$\textbackslash{}varphi(x, y, z) =
(x, y, 2 - z)\$, a composição \$\textbackslash{}varphi
\textbackslash{}circ \textbackslash{}pi\^{}\{-1\}\$ também parametriza
outra vizinhança da esfera. Em particular \$\textbackslash{}varphi
\textbackslash{}circ \textbackslash{}pi\^{}\{-1\} (0, 0) = (0, 0, 4)\$,
que era o único ponto que tinha ficado faltando na parametrização
anterior.

\hypertarget{questuxe3o-17-}{%
\subsubsection{\texorpdfstring{Questão 17
🖋️\protect\hyperlink{questuxe3o-17-}{\#}}{Questão 17 🖋️\#}}\label{questuxe3o-17-}}

Defina uma curva regular analogamente a uma superfície regular. Prove
que

\hypertarget{resoluuxe7uxe3o-9}{%
\paragraph{Resolução}\label{resoluuxe7uxe3o-9}}

Um subconjunto \$C \textbackslash{}subset R\^{}3\$ é uma \emph{curva
regular} se, para todo \$p \textbackslash{}in C\$, existir uma
vizinhança \$V\$ em \$R\^{}3\$ e uma aplicação
\$\textbackslash{}mathbf\{x\}: I \textbackslash{}rightarrow V
\textbackslash{}cap C\$ de um \emph{intervalo aberto} \$I
\textbackslash{}subset R\$ em \$V \textbackslash{}cap C
\textbackslash{}subset R\^{}3\$ tal que

\begin{enumerate}
\tightlist
\item
  \$\textbackslash{}mathbf\{x\}\$ é diferenciável
\item
  \$\textbackslash{}mathbf\{x\}\$ é homeomorfismo
\item
  para todo \$t \textbackslash{}in I\$ temos
  \$\textbackslash{}mathbf\{x\}' \textbackslash{}neq 0\$
\end{enumerate}

\textbf{a)} A imagem inversa de um valor regular de uma função
diferenciável

\$\$f: U \textbackslash{}subset R\^{}2 \textbackslash{}rightarrow R\$\$

é uma curva plana regular. Dê um exemplo de tal curva que não é conexa.

\hypertarget{resoluuxe7uxe3o-10}{%
\paragraph{Resolução}\label{resoluuxe7uxe3o-10}}

Suponha que \$u \textbackslash{}in R\$ é um valor regular de \$f\$ e
seja \$p = (x\_0, y\_0)\$ um ponto de \$f\^{}\{-1\}(p)\$. Sabemos que
\$f'(u) \textbackslash{}neq 0\$, então alguma das coordenadas dessa
derivada seria diferente de \$0\$. Sem perda de generalizada podemos
assumir que é o \$y\$. Vamos definir uma \$F : R\^{}2
\textbackslash{}rightarrow R\^{}2\$ tal que \$F(x, y) = (x, f(x, y))\$.
A derivada de \$F\$ é dada por \$\$\textbackslash{}mathrm\{d\}F\_p =
\textbackslash{}begin\{pmatrix\} 1 \& 0
\textbackslash{}\textbackslash{}\\
f\_x \& f\_y \textbackslash{}\textbackslash{}\\
\textbackslash{}end\{pmatrix\}\$\$

Como \$f\_y \textbackslash{}neq 0\$ por hipótese, esse determinante é
diferente de \$0\$. Assim podemos aplicar o teorema da função inversa
para obter uma \$F\^{}\{-1\}\$ numa vizinhança de \$p\$. Em particular,
em \$f\^{}\{-1\}(u)\$ podemos definir a função \$\textbackslash{}varphi:
x \textbackslash{}mapsto F\^{}\{-1\}(x, u)\$ que vai parametrizar os
pontos da vizinhança em função do \$x\$.

Um exemplo de curva regular que não é conexa é \$y\^{}2 - x\^{}2 = -1\$,
dá uma hipérbole. Ela possui os pontos \$(1, 0)\$ e \$(-1, 0)\$, mas não
possui nenhum ponto com \$x \textbackslash{}in (-1, 1)\$

\textbf{b)} A imagem inversa de um valor regular de uma aplicação
diferenciável

\$\$F: U \textbackslash{}subset R\^{}3 \textbackslash{}rightarrow
R\^{}2\$\$

é uma curva regular em R\^{}3. Mostre a relação entre esta proposição e
a maneira clássica de definir a curva em \$R\^{}3\$ como uma intersecção
de duas superfícies.

\hypertarget{resoluuxe7uxe3o-11}{%
\paragraph{Resolução}\label{resoluuxe7uxe3o-11}}

A ideia é bem parecida com a da questão anterior, mas dessa vez vamos
definir a função \$G(x, y, z) = (x, F\_1(x, y, z), F\_2(x, y, z))\$ para
aplicarmos o teorema da função inversa. Feito isso a função
\$\textbackslash{}varphi(x) = G\^{}\{-1\}(x, p\_1, p\_2)\$ irá
parametrizar os pontos da curva.

A analogia vem do fato de que, se temos \$p = (p\_1, p\_2)\$ um valor
regular de \$F\$, definirmos \$F\_1 = \textbackslash{}pi\_1
\textbackslash{}circ F\$ e \$F\_2 = \textbackslash{}pi\_2
\textbackslash{}circ F\$, então \$p\_1\$ é um valor regular de \$F\_1\$
e \$p\_2\$ é um valor regular de \$F\_2\$. Logo \$F\_1\^{}\{-1\}(p\_1)\$
e \$F\_2\^{}\{-1\}(p\_2)\$ são superfícies regulares, e a intersecção
delas é a curva regular \$F\^{}\{-1\}(p)\$.

\textbf{c)} O conjunto \$C = \{(x, y) \textbackslash{}in
R\^{}2;\textbackslash{} x\^{}2 = y\^{}3\}\$ não é uma curva regular.

\hypertarget{resoluuxe7uxe3o-12}{%
\paragraph{Resolução}\label{resoluuxe7uxe3o-12}}

Suponha que tivéssemos uma \$\textbackslash{}varphi:
(-\textbackslash{}varepsilon, \textbackslash{}varepsilon)
\textbackslash{}rightarrow R\^{}2\$ que parametrizasse uma vizinhança do
\$0\$ de forma que \$\textbackslash{}varphi(0) = (0, 0)\$ e

\$\$t\^{}2 = \textbackslash{}varphi(t)\^{}3\$\$

derivando ambos os lados duas vezes

\$\$2 \textbackslash{}cdot t = 3 \textbackslash{}cdot
\textbackslash{}varphi(t)\^{}2 \textbackslash{}varphi'(t)\$\$

\$\$2 = 3 \textbackslash{}cdot (2\textbackslash{}varphi(t)
\textbackslash{}varphi'(t)\^{}2 + \textbackslash{}varphi(t)\^{}2
\textbackslash{}varphi''(t))\$\$ \$\$2 = 3 \textbackslash{}varphi(t)
\textbackslash{}cdot (2\textbackslash{}varphi'(t)\^{}2 +
\textbackslash{}varphi(t) \textbackslash{}varphi''(t))\$\$

mas \$\textbackslash{}varphi(0) = 0\$, absurdo. Daí que a curva não pode
ser regular.

\hypertarget{capuxedtulo-2---seuxe7uxe3o-3}{%
\subsection{\texorpdfstring{Capítulo 2 - Seção
3\protect\hyperlink{capuxedtulo-2---seuxe7uxe3o-3}{\#}}{Capítulo 2 - Seção 3\#}}\label{capuxedtulo-2---seuxe7uxe3o-3}}

\hypertarget{questuxe3o-6}{%
\subsubsection{\texorpdfstring{Questão
6\protect\hyperlink{questuxe3o-6}{\#}}{Questão 6\#}}\label{questuxe3o-6}}

Prove que a definição de aplicação diferenciável entre superfícies não
depende das parametrizações escolhidas.

\begin{itemize}
\tightlist
\item
  \href{https://impression28.github.io/tags/matematica/}{matematica}
\item
  \href{https://impression28.github.io/tags/geometria-diferencial/}{geometria-diferencial}
\end{itemize}

\href{https://impression28.github.io/exercicios/lista1-tal/}{{Previous
Post} {Lista 1 --- Tópicos de Álgebra Linear}}

\href{https://github.com/impression28}{}

\href{https://twitter.com/impression28}{}

\begin{itemize}
\tightlist
\item
  \protect\hyperlink{capuxedtulo-1--seuxe7uxe3o-3}{Capítulo 1 --- Seção
  3}

  \begin{itemize}
  \tightlist
  \item
    \protect\hyperlink{questuxe3o-8-}{Questão 8 🖋️}
  \item
    \protect\hyperlink{questuxe3o-9-}{Questão 9 🖋️}
  \item
    \protect\hyperlink{questuxe3o-10-}{Questão 10 🖋️}
  \end{itemize}
\item
  \protect\hyperlink{capuxedtulo-1--seuxe7uxe3o-5}{Capítulo 1 --- Seção
  5}

  \begin{itemize}
  \tightlist
  \item
    \protect\hyperlink{questuxe3o-4-}{Questão 4 🖋️}
  \item
    \protect\hyperlink{questuxe3o-6-}{Questão 6 🖋️}
  \item
    \protect\hyperlink{questuxe3o-12}{Questão 12}
  \end{itemize}
\item
  \protect\hyperlink{capuxedtulo-2--seuxe7uxe3o-2}{Capítulo 2 --- Seção
  2}

  \begin{itemize}
  \tightlist
  \item
    \protect\hyperlink{questuxe3o-16-}{Questão 16 🖋️}
  \item
    \protect\hyperlink{questuxe3o-17-}{Questão 17 🖋️}
  \end{itemize}
\item
  \protect\hyperlink{capuxedtulo-2---seuxe7uxe3o-3}{Capítulo 2 - Seção
  3}

  \begin{itemize}
  \tightlist
  \item
    \protect\hyperlink{questuxe3o-6}{Questão 6}
  \end{itemize}
\end{itemize}
